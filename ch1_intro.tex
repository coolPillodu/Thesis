\chapter{Introduction}
\section{Overview}
A Smart Metering System (SMS) is composed of many Smart Meters (SM), devices that measure the consumption or production of commodities like electrical energy, gas, or water in a physical metrological measurement metering unit and transform the measured values into digital information, that can eventually be forwarded to an accounting system, and further to a billing system. A data collector  called a Gateway is used for the collection of the metered data from the SMs and  forward the data to the accounting systems operated by one enterprise, which then delivers the accounting information to another enterprise. For example, a power distribution operator  might collect the accounting information and provide it in aggregated form to the power supply company. Since recently\footnote{March, 2011} there is a Protection Profile (PP) for the Gateway proposed by the German Federal Office for Information Security (BSI) \cite{BSI_pp} that sets out the security requirement for the SM networks in Germany and in this process defines a reference network architecture.

The communication between the SMs and the Gateway is unmanaged, requiring self-configuration support, confidentiality of the communicated information, and may need to take place in a hostile environment. Whereas the communication between the Gateways and the accounting system is managed, requires integrity and confidentiality of the communicated accumulated information, and access to the accounting system by the power generation companies is regulated.

The idea is now to utilize HIP \cite{HIP_rfc} to initiate IPsec protected communication, such that asymmetric keys, generated on smart meters, are used as identifier and that the Gateway accepts the meter as its slave, according to the Resurrecting Duckling security policy \cite{Sta_duck}. It is assumed that the Gateway can be identified to the accounting system by asymmetric keying material managed by means of a Public Key Interface (PKI).


\section{Problem statement}
The aim of this thesis project is to  enhance network security domains so that communication in future smart metering provides secure auto-configurable HIP-enable networks with  a self-generated cryptographic identifier so that a SM can communicate via a Gateway.

The thesis project can be divided into the following three phases:
\begin{itemize}
\item Literature study: to study the different network architectures, communication protocols and technologies suitable for smart meters. 
\item Designing: this phase encompasses the overall design of a demo network scenario suitable for implementation.
\item Implementation: to implement the proof-of-concept in a lab environment and to validate/test the same against security threats and performance.
\end{itemize}