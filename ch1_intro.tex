\chapter{Introduction}
\section{Overview}
A Smart Metering System (SMS) is composed of many Smart Meters (SM), devices that measure the consumption or production of commodities like electrical energy, gas, or water in a physical metrological measurement metering unit and transform the measured values into digital information, that can eventually be forwarded to an accounting system, and further to a billing system. 

\fxnote*{Needs paraphrasing


}{The adoption of Smart Meters can contribute to reduced carbon dioxide emissions in two ways. Firstly, they allow showing customers their electricity consumption in an illustrative manner, which can be a motivation for saving energy. Secondly, SM enables dynamic pricing dependent on the current supply situation, making it easier to shift electricity consumptions to times when renewable energy source (e.g., wind energy) are available. This way, less peak-load electricity generation plants-which tend to be inefficient and often based on fossil fuels-are required.}

Smart Meters are capable of real-time monitoring of a commodity consumption and thus provide detailed profile of the commodity usage per customer. The usage profile can then be used for optimizing the usage of the commodity. For example, electricity pricing usually peaks at certain predictable times of the day and the season and thus in some countries electricity billing is by the time of day usage. By utilizing the consumption profile generated by the SM, it is possible to adapt the electricity usage in a home accordingly, such that less power is consumed during the peak (high price) hours.

Although real time monitoring is advantageous there are security threats associated. It is possible to \fxnote*{ cite security threats from real time monitoring}{[]}.
%\cite{}
Thus, security of Smart Meters represent a worst-case scenario: the devices lack sufficient power to execute strong security software; they are placed in physically non-secure location and  

The communication between the SMs and the Gateway is unmanaged, requiring self-configuration support, confidentiality of the communicated information, and may need to take place in a hostile environment. Whereas the communication between the Gateways and the accounting system is managed, requires integrity and confidentiality of the communicated accumulated information, and access to the accounting system by the power generation companies is regulated.

\section{Problem statement}
The aim of this thesis project is to  enhance network security domains so that communication in future smart metering provides secure auto-configurable networks.

The thesis project can be divided into the following three phases:
\begin{enumerate}
\item Evaluting different alternatives for realizing the network architecture and security requirements.
\item Designing a system architecture and a demonstration scenario wherein the network could be emulated in a lab environment.
\item Realizing a proof-of-concept network and analyizing the performance metrics and the security threats of the network.
\end{enumerate}